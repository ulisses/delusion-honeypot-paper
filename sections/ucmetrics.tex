\subsection{Use Case Metrics}

Use Cases Diagrams are graphical representations of entities which interact with the system (\emph{actors}) and operations that the system must perform for them.
They define a sequence of actions which illustrate a specific way of using the system.

These diagrams are functional specifications, collected at the beginning of a system development process.
They are crucial to an early estimate of the system complexity and its development efforts, as we can see by the UC metrics defined in several works like~\cite{Kim02developingsoftware},\cite{Mohagheghi05effortestimation}, \cite{Ribu01estimatingobject-oriented}.

In fact, measuring the number of Use Cases, actors and communications among them is a good indicator of the system complexity, as well as to quantify the relationship between diagrams (i.e. estimates the number of UC that extend or include others).

One remarkable work on this area was performed by Michele Marchesi~\cite{Marchesi:1998:OMU:522081.795010}.
Table~\ref{t:ucm} illustrates the Use Case metrics defined on this work. 
 
\begin{table}[h]\centering
\begin{tabular}{ p{1,5cm} | p{10.5cm}}
\multicolumn{2}{l}{\textbf{Marchesi Metrics}} \\ \hline
\textbf{Metric} & \textbf{Description} \\ \hline
NA & Number of actors of the system. \\ \hline
UC1 & Number of Use Cases in the system. \\ \hline 
UC2 & Number of communications among UC and Actors  \\ \hline 
UC3 & Number of communications among UC and Actores without redundancies \\ \hline 
UC4 & Global complexity of the system \\ \hline 
\end{tabular}
\caption{\small{Marchesi Use Case Metrics}}
\label{t:ucm}
\end{table}

The \textbf{UC4} metric represents a balance of the global complexity of the system, and its value is obtained through the values of \textbf{UC1}, \textbf{UC2} and \textbf{UC3} metrics.
